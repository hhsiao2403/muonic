% Generated by Sphinx.
\def\sphinxdocclass{report}
\documentclass[letterpaper,10pt,english]{sphinxmanual}
\usepackage[utf8]{inputenc}
\DeclareUnicodeCharacter{00A0}{\nobreakspace}
\usepackage[T1]{fontenc}
\usepackage{babel}
\usepackage{times}
\usepackage[Bjarne]{fncychap}
\usepackage{longtable}
\usepackage{sphinx}
\usepackage{multirow}


\title{muonic Documentation}
\date{December 19, 2012}
\release{0.1}
\author{robert.franke,achim.stoessl}
\newcommand{\sphinxlogo}{}
\renewcommand{\releasename}{Release}
\makeindex

\makeatletter
\def\PYG@reset{\let\PYG@it=\relax \let\PYG@bf=\relax%
    \let\PYG@ul=\relax \let\PYG@tc=\relax%
    \let\PYG@bc=\relax \let\PYG@ff=\relax}
\def\PYG@tok#1{\csname PYG@tok@#1\endcsname}
\def\PYG@toks#1+{\ifx\relax#1\empty\else%
    \PYG@tok{#1}\expandafter\PYG@toks\fi}
\def\PYG@do#1{\PYG@bc{\PYG@tc{\PYG@ul{%
    \PYG@it{\PYG@bf{\PYG@ff{#1}}}}}}}
\def\PYG#1#2{\PYG@reset\PYG@toks#1+\relax+\PYG@do{#2}}

\expandafter\def\csname PYG@tok@gd\endcsname{\def\PYG@tc##1{\textcolor[rgb]{0.63,0.00,0.00}{##1}}}
\expandafter\def\csname PYG@tok@gu\endcsname{\let\PYG@bf=\textbf\def\PYG@tc##1{\textcolor[rgb]{0.50,0.00,0.50}{##1}}}
\expandafter\def\csname PYG@tok@gt\endcsname{\def\PYG@tc##1{\textcolor[rgb]{0.00,0.25,0.82}{##1}}}
\expandafter\def\csname PYG@tok@gs\endcsname{\let\PYG@bf=\textbf}
\expandafter\def\csname PYG@tok@gr\endcsname{\def\PYG@tc##1{\textcolor[rgb]{1.00,0.00,0.00}{##1}}}
\expandafter\def\csname PYG@tok@cm\endcsname{\let\PYG@it=\textit\def\PYG@tc##1{\textcolor[rgb]{0.25,0.50,0.56}{##1}}}
\expandafter\def\csname PYG@tok@vg\endcsname{\def\PYG@tc##1{\textcolor[rgb]{0.73,0.38,0.84}{##1}}}
\expandafter\def\csname PYG@tok@m\endcsname{\def\PYG@tc##1{\textcolor[rgb]{0.13,0.50,0.31}{##1}}}
\expandafter\def\csname PYG@tok@mh\endcsname{\def\PYG@tc##1{\textcolor[rgb]{0.13,0.50,0.31}{##1}}}
\expandafter\def\csname PYG@tok@cs\endcsname{\def\PYG@tc##1{\textcolor[rgb]{0.25,0.50,0.56}{##1}}\def\PYG@bc##1{\setlength{\fboxsep}{0pt}\colorbox[rgb]{1.00,0.94,0.94}{\strut ##1}}}
\expandafter\def\csname PYG@tok@ge\endcsname{\let\PYG@it=\textit}
\expandafter\def\csname PYG@tok@vc\endcsname{\def\PYG@tc##1{\textcolor[rgb]{0.73,0.38,0.84}{##1}}}
\expandafter\def\csname PYG@tok@il\endcsname{\def\PYG@tc##1{\textcolor[rgb]{0.13,0.50,0.31}{##1}}}
\expandafter\def\csname PYG@tok@go\endcsname{\def\PYG@tc##1{\textcolor[rgb]{0.19,0.19,0.19}{##1}}}
\expandafter\def\csname PYG@tok@cp\endcsname{\def\PYG@tc##1{\textcolor[rgb]{0.00,0.44,0.13}{##1}}}
\expandafter\def\csname PYG@tok@gi\endcsname{\def\PYG@tc##1{\textcolor[rgb]{0.00,0.63,0.00}{##1}}}
\expandafter\def\csname PYG@tok@gh\endcsname{\let\PYG@bf=\textbf\def\PYG@tc##1{\textcolor[rgb]{0.00,0.00,0.50}{##1}}}
\expandafter\def\csname PYG@tok@ni\endcsname{\let\PYG@bf=\textbf\def\PYG@tc##1{\textcolor[rgb]{0.84,0.33,0.22}{##1}}}
\expandafter\def\csname PYG@tok@nl\endcsname{\let\PYG@bf=\textbf\def\PYG@tc##1{\textcolor[rgb]{0.00,0.13,0.44}{##1}}}
\expandafter\def\csname PYG@tok@nn\endcsname{\let\PYG@bf=\textbf\def\PYG@tc##1{\textcolor[rgb]{0.05,0.52,0.71}{##1}}}
\expandafter\def\csname PYG@tok@no\endcsname{\def\PYG@tc##1{\textcolor[rgb]{0.38,0.68,0.84}{##1}}}
\expandafter\def\csname PYG@tok@na\endcsname{\def\PYG@tc##1{\textcolor[rgb]{0.25,0.44,0.63}{##1}}}
\expandafter\def\csname PYG@tok@nb\endcsname{\def\PYG@tc##1{\textcolor[rgb]{0.00,0.44,0.13}{##1}}}
\expandafter\def\csname PYG@tok@nc\endcsname{\let\PYG@bf=\textbf\def\PYG@tc##1{\textcolor[rgb]{0.05,0.52,0.71}{##1}}}
\expandafter\def\csname PYG@tok@nd\endcsname{\let\PYG@bf=\textbf\def\PYG@tc##1{\textcolor[rgb]{0.33,0.33,0.33}{##1}}}
\expandafter\def\csname PYG@tok@ne\endcsname{\def\PYG@tc##1{\textcolor[rgb]{0.00,0.44,0.13}{##1}}}
\expandafter\def\csname PYG@tok@nf\endcsname{\def\PYG@tc##1{\textcolor[rgb]{0.02,0.16,0.49}{##1}}}
\expandafter\def\csname PYG@tok@si\endcsname{\let\PYG@it=\textit\def\PYG@tc##1{\textcolor[rgb]{0.44,0.63,0.82}{##1}}}
\expandafter\def\csname PYG@tok@s2\endcsname{\def\PYG@tc##1{\textcolor[rgb]{0.25,0.44,0.63}{##1}}}
\expandafter\def\csname PYG@tok@vi\endcsname{\def\PYG@tc##1{\textcolor[rgb]{0.73,0.38,0.84}{##1}}}
\expandafter\def\csname PYG@tok@nt\endcsname{\let\PYG@bf=\textbf\def\PYG@tc##1{\textcolor[rgb]{0.02,0.16,0.45}{##1}}}
\expandafter\def\csname PYG@tok@nv\endcsname{\def\PYG@tc##1{\textcolor[rgb]{0.73,0.38,0.84}{##1}}}
\expandafter\def\csname PYG@tok@s1\endcsname{\def\PYG@tc##1{\textcolor[rgb]{0.25,0.44,0.63}{##1}}}
\expandafter\def\csname PYG@tok@gp\endcsname{\let\PYG@bf=\textbf\def\PYG@tc##1{\textcolor[rgb]{0.78,0.36,0.04}{##1}}}
\expandafter\def\csname PYG@tok@sh\endcsname{\def\PYG@tc##1{\textcolor[rgb]{0.25,0.44,0.63}{##1}}}
\expandafter\def\csname PYG@tok@ow\endcsname{\let\PYG@bf=\textbf\def\PYG@tc##1{\textcolor[rgb]{0.00,0.44,0.13}{##1}}}
\expandafter\def\csname PYG@tok@sx\endcsname{\def\PYG@tc##1{\textcolor[rgb]{0.78,0.36,0.04}{##1}}}
\expandafter\def\csname PYG@tok@bp\endcsname{\def\PYG@tc##1{\textcolor[rgb]{0.00,0.44,0.13}{##1}}}
\expandafter\def\csname PYG@tok@c1\endcsname{\let\PYG@it=\textit\def\PYG@tc##1{\textcolor[rgb]{0.25,0.50,0.56}{##1}}}
\expandafter\def\csname PYG@tok@kc\endcsname{\let\PYG@bf=\textbf\def\PYG@tc##1{\textcolor[rgb]{0.00,0.44,0.13}{##1}}}
\expandafter\def\csname PYG@tok@c\endcsname{\let\PYG@it=\textit\def\PYG@tc##1{\textcolor[rgb]{0.25,0.50,0.56}{##1}}}
\expandafter\def\csname PYG@tok@mf\endcsname{\def\PYG@tc##1{\textcolor[rgb]{0.13,0.50,0.31}{##1}}}
\expandafter\def\csname PYG@tok@err\endcsname{\def\PYG@bc##1{\setlength{\fboxsep}{0pt}\fcolorbox[rgb]{1.00,0.00,0.00}{1,1,1}{\strut ##1}}}
\expandafter\def\csname PYG@tok@kd\endcsname{\let\PYG@bf=\textbf\def\PYG@tc##1{\textcolor[rgb]{0.00,0.44,0.13}{##1}}}
\expandafter\def\csname PYG@tok@ss\endcsname{\def\PYG@tc##1{\textcolor[rgb]{0.32,0.47,0.09}{##1}}}
\expandafter\def\csname PYG@tok@sr\endcsname{\def\PYG@tc##1{\textcolor[rgb]{0.14,0.33,0.53}{##1}}}
\expandafter\def\csname PYG@tok@mo\endcsname{\def\PYG@tc##1{\textcolor[rgb]{0.13,0.50,0.31}{##1}}}
\expandafter\def\csname PYG@tok@mi\endcsname{\def\PYG@tc##1{\textcolor[rgb]{0.13,0.50,0.31}{##1}}}
\expandafter\def\csname PYG@tok@kn\endcsname{\let\PYG@bf=\textbf\def\PYG@tc##1{\textcolor[rgb]{0.00,0.44,0.13}{##1}}}
\expandafter\def\csname PYG@tok@o\endcsname{\def\PYG@tc##1{\textcolor[rgb]{0.40,0.40,0.40}{##1}}}
\expandafter\def\csname PYG@tok@kr\endcsname{\let\PYG@bf=\textbf\def\PYG@tc##1{\textcolor[rgb]{0.00,0.44,0.13}{##1}}}
\expandafter\def\csname PYG@tok@s\endcsname{\def\PYG@tc##1{\textcolor[rgb]{0.25,0.44,0.63}{##1}}}
\expandafter\def\csname PYG@tok@kp\endcsname{\def\PYG@tc##1{\textcolor[rgb]{0.00,0.44,0.13}{##1}}}
\expandafter\def\csname PYG@tok@w\endcsname{\def\PYG@tc##1{\textcolor[rgb]{0.73,0.73,0.73}{##1}}}
\expandafter\def\csname PYG@tok@kt\endcsname{\def\PYG@tc##1{\textcolor[rgb]{0.56,0.13,0.00}{##1}}}
\expandafter\def\csname PYG@tok@sc\endcsname{\def\PYG@tc##1{\textcolor[rgb]{0.25,0.44,0.63}{##1}}}
\expandafter\def\csname PYG@tok@sb\endcsname{\def\PYG@tc##1{\textcolor[rgb]{0.25,0.44,0.63}{##1}}}
\expandafter\def\csname PYG@tok@k\endcsname{\let\PYG@bf=\textbf\def\PYG@tc##1{\textcolor[rgb]{0.00,0.44,0.13}{##1}}}
\expandafter\def\csname PYG@tok@se\endcsname{\let\PYG@bf=\textbf\def\PYG@tc##1{\textcolor[rgb]{0.25,0.44,0.63}{##1}}}
\expandafter\def\csname PYG@tok@sd\endcsname{\let\PYG@it=\textit\def\PYG@tc##1{\textcolor[rgb]{0.25,0.44,0.63}{##1}}}

\def\PYGZbs{\char`\\}
\def\PYGZus{\char`\_}
\def\PYGZob{\char`\{}
\def\PYGZcb{\char`\}}
\def\PYGZca{\char`\^}
\def\PYGZam{\char`\&}
\def\PYGZlt{\char`\<}
\def\PYGZgt{\char`\>}
\def\PYGZsh{\char`\#}
\def\PYGZpc{\char`\%}
\def\PYGZdl{\char`\$}
\def\PYGZti{\char`\~}
% for compatibility with earlier versions
\def\PYGZat{@}
\def\PYGZlb{[}
\def\PYGZrb{]}
\makeatother

\begin{document}

\maketitle
\tableofcontents
\phantomsection\label{index::doc}



\chapter{Documentation for release 0.1, documentation built on December 19, 2012}
\label{index:welcome-to-muonic-documentation}\label{index:documentation-for-release-release-documentation-built-on-today}
Contents:


\section{muonic - a python gui for QNET experiments}
\label{intro::doc}\label{intro:muonic-a-python-gui-for-qnet-experiments}
The muonic project provides an interface to communicate with QNet DAQ cards and to perform simple analysis of QNet data.
Its goal is to ensure easy and stable access to the QNet cards and visualize some of the features of the cards. It is meant to be used in school projects, so it should be easy to use even by people who do not have lots of programming or LINUX backround. Automated data taking can be used to ensure no valuable data is lost.


\subsection{Licence and terms of agreement}
\label{intro:licence-and-terms-of-agreement}
Muonic is ditributet under the terms of GPL (Gnu Public License). With the use of the software you accept the condidions of the GPL. This means also that the authors can not be made responsible for any damage of any kind to hard- or software.


\section{muonic setup and installation}
\label{setup::doc}\label{setup:muonic-setup-and-installation}
Muonic consists of two main parts:
1. the python package \emph{muonic}
2. a python executable


\subsection{prerequesitories}
\label{setup:prerequesitories}
muonic needs the following packages to be installed (list may not be complete!)
\begin{itemize}
\item {} 
python-scipy

\item {} 
python-matplotlib

\item {} 
python-numpy

\item {} 
python-qt4

\item {} 
python-serial

\end{itemize}


\subsection{installation with the setup.py script}
\label{setup:installation-with-the-setup-py-script}
Run the following command in the muonic main directory

\emph{python setup.py install}

This will put the muonic package into your python site-packages directory and alsot the exectuables \emph{muonic} and \emph{which\_tty\_daq} to your user/bin directory.

The use of python-virtualenv is recommended.


\subsection{installing muonic without the setup script}
\label{setup:installing-muonic-without-the-setup-script}
You just need the script \emph{./bin/muonic} to the upper directory and rename it to \emph{muonic.py}.
You can do this by typing

\emph{mv bin/muonic muonic.py}

while being in the muonic main directory.

Afterwards you have to create the folder \emph{muonic\_data} in your home directory.

\emph{mkdir \textasciitilde{}/muonic\_data}


\section{How to use muonic}
\label{tutorial::doc}\label{tutorial:how-to-use-muonic}

\subsection{start muonic}
\label{tutorial:start-muonic}
If you have setup muonic via the provided setup.py script or if you hav put the package somewhere in your PYTHONPATH, simple call from the terminal

\code{muonic {[}OPTIONS{]} xy}

where \code{xy} are two characters which you can choose freely. You will find this two letters occuring in automatically generated files, so that you can identify them.

For help you can call

\code{muonic -{-}help}

{[}OPTIONS{]}
\index{muonic command line option!-s}\index{-s!muonic command line option}

\begin{fulllineitems}
\phantomsection\label{tutorial:cmdoption-muonic-s}\pysigline{\bfcode{-s}\code{}}\pysigline{\bfcode{use~the~simulation~mode~of~muonic.~This~should~only~used~for~testing~and~developing~the~software}}
\end{fulllineitems}

\index{muonic command line option!-d}\index{-d!muonic command line option}

\begin{fulllineitems}
\phantomsection\label{tutorial:cmdoption-muonic-d}\pysigline{\bfcode{-d}\code{}}\pysigline{\bfcode{debug~mode.~Use~it~to~generate~more~log~messages~on~the~console.}}
\end{fulllineitems}

\index{muonic command line option!-t sec}\index{-t sec!muonic command line option}

\begin{fulllineitems}
\phantomsection\label{tutorial:cmdoption-muonic-t}\pysigline{\bfcode{-t}\code{~sec}}\pysigline{\bfcode{change~the~timewindow~for~the~calculation~of~the~rates.~If~you~expect~very~low~rates,~you~might~consider~to~change~it~to~larger~values.}}\pysigline{\bfcode{default~is~5~seconds.}}
\end{fulllineitems}

\index{muonic command line option!-p}\index{-p!muonic command line option}

\begin{fulllineitems}
\phantomsection\label{tutorial:cmdoption-muonic-p}\pysigline{\bfcode{-p}\code{}}\pysigline{\bfcode{automatically~write~a~file~with~pulsetimes~in~a~non~hexadecimal~representation}}
\end{fulllineitems}

\index{muonic command line option!-n}\index{-n!muonic command line option}

\begin{fulllineitems}
\phantomsection\label{tutorial:cmdoption-muonic-n}\pysigline{\bfcode{-n}\code{}}\pysigline{\bfcode{supress~any~status~messages~in~the~output~raw~data~file,~might~be~useful~if~you~want~use~muonic~only~for~data~taking~and~use~another~script~afterwards~for~analysis.}}
\end{fulllineitems}



\subsection{Saving files with muonic}
\label{tutorial:saving-files-with-muonic}
All files which are saved by muonic are ASCII files. The filenames are as follows:

\begin{notice}{warning}{Warning:}
currently all files are saved under \$HOME/muonic\_data. This directory must exist. If you use the provided setup script, it is created automatically
\end{notice}

\emph{YYYY-MM-DD\_HH-MM-SS\_TYPE\_MEASUREMENTTME\_xy}
\begin{itemize}
\item {} 
\emph{YYYY-MM-DD} is the date of the measurement start

\item {} 
\emph{HH-MM-SS} is the GMT time of the measurement start

\item {} 
\emph{MEASUREMENTTIME} if muonic is closed, each file gets is corresponding measurement time (in hours) assigned.

\item {} 
\emph{xy} the two letters which were specified at the start of muonic

\item {} 
\emph{TYPE} might be one of the following:

\end{itemize}
\begin{itemize}
\item {} 
\emph{RAW} the raw ASCII output of the DAQ card, this is only saved if the `Save to file' button in clicked in the `Daq output' window of muonic

\item {} 
\emph{R} is an automatically saved ASCII file which contains the rate measurement data, this can then be used to plot with e.g. gnuplot later on

\item {} 
\emph{L} specifies a file with times of registered muon decays. This file is automatically saved if a muon decay measurement is started.

\item {} 
\emph{P} stands for a file which contains a non-hex representation of the registered pulses. This file is only save if the \emph{-p} option is given at the start of muonic

\end{itemize}

Representation of the pulses:

\emph{(69.15291364, {[}(0.0, 12.5){]}, {[}(2.5, 20.0){]}, {[}{]}, {[}{]})}

This is a python-tuple which contains the triggertime of the event and four lists with more tuples. The lists represent the channels (0-3 from left to right) and each tuple stands for a leading and a falling edge of a registered pulse. To get the exact time of the pulse start, one has to add the pulse LE and FE times to the triggertime

\begin{notice}{note}{Note:}
For calculation of the LE and FE pulse times a TMC is used. It seems that for some DAQs cards a TMC bin is 1.25 ns wide, allthough the documentation says something else.
The triggertime is calculated using a CPLD which runs in some cards at 25MHz, which gives a binwidth of the CPLD time of 40 ns.
Please keep this limited precision in mind when adding CPLD and TMC times.
\end{notice}


\subsection{Performing measurements with muonic}
\label{tutorial:performing-measurements-with-muonic}

\subsubsection{Setting up the DAQ}
\label{tutorial:setting-up-the-daq}
For DAQ setup it is recommended to use the `settings' menu, allthough everything can also be setup via the command line in the DAQ output window (see below.)
Muonic translates the chosen settings to the corresponding DAQ commands and sends them to the DAQ. So if you want to change things like the coincidence time window, you have to issue the corresponding DAQ command in the DAQ output window.

Two menu items are of interest here:
* Channel Configuration: Enable the channels here and set coincidence settings. A veto channel can also be specified.

\begin{notice}{note}{Note:}
You have to ensure that the checkboxes for the channels you want to use are checked before you leave this dialogue, otherwise the channel gets deactivated.
\end{notice}

\begin{notice}{note}{Note:}
The concidence is realized by the DAQ in a way that no specific channels can be given. Instead this is meant as an `any' condition.
So `twofold' means that `any two of the enabled channels' must claim signal instead of two specific ones (like 1 and 2).
\end{notice}

\begin{notice}{warning}{Warning:}
Measurements ad DESY indicated that the veto feature of the DAq might not work properly in all cases.
\end{notice}
\begin{itemize}
\item {} 
Thresholds: For each channel a threshold (in milliVolts) can be specified. Pulse which are below this threshold are rejected. Use this for electronic noise supression.

\end{itemize}

\begin{notice}{note}{Note:}
A proper calibration of the individual channels is the key to a succesfull measurement!
\end{notice}


\subsubsection{Looking at raw DAQ data}
\label{tutorial:looking-at-raw-daq-data}
The first tab of muonic displays the raw ASCII DAQ data.
This can be saved to a file. If the DAQ status messages should be supressed in that file, the option \emph{-n} should be given at the start of muonic.
The edit field can be used to send messages to the DAQ. For an overview over the messages, look here (link not available yet!).
To issue such an command periodically, you can use the button `Periodic Call'

\begin{notice}{note}{Note:}
The two most importand DAQ commands are `CD' (`counter disable') and `CE' (`counter enable'). Pulse information is only given out by the DAQ if the counter is set to enabled. All pulse related features may not work properly if the counter is set to disabled.
\end{notice}


\subsubsection{Muon Rates}
\label{tutorial:muon-rates}
In this tab a plot of the measured muonrates is displayed. A triggerrate is only shown if a coincidence condition is set.
In the block on the left side of the tab the average rates are displayed since the measurement start. Below you can find the number of counts for the individual channels. The measurement can be reset by clicking on `Restart'. The `Stop' button can be used to temporarily hold the plot to have a better look at it.

\begin{notice}{note}{Note:}
You can use the displayed `max rate' at the left bottom to check if anything with the measurement went wrong.
\end{notice}

\begin{notice}{note}{Note:}
Currently the plot shows only the last 200 seconds. If you want to have a longer timerange, you can use the information which is automatically stored in the `R' file.(see above)
\end{notice}


\subsubsection{Muon Lifetime}
\label{tutorial:muon-lifetime}
A lifetime measurement of muons can be performed here. A histogram of time differences between succeding pulses in the same channel is shown. It can be fit with an exponential by clicking on `Fit!'. The fit lifetime is then shown in the above right of the plot, for an estimate on the errors you have to look at the console.

\begin{notice}{warning}{Warning:}
This feature might not work properly, especially when used with the standard scintilators! Use it with extreme care.
\end{notice}


\subsubsection{Pulse Analyzer}
\label{tutorial:pulse-analyzer}
You can have a look at the pulsewidhts in this plot. The height of the pulses is lost during the digitization prozess, so all pulses have the same height here.


\section{Fermilab DAQ - hardware documentation}
\label{hardware:fermilab-daq-hardware-documentation}\label{hardware::doc}

\subsection{ASCII DAQ output format}
\label{hardware:ascii-daq-output-format}
sample line of DAQ output - example for the daq data format

\begin{tabulary}{\linewidth}{|L|L|L|L|L|L|L|L|L|L|L|L|L|L|L|L|}
\hline
\textbf{
triggers
} & \textbf{
r0
} & \textbf{
f0
} & \textbf{
r1
} & \textbf{
f1
} & \textbf{
r2
} & \textbf{
f2
} & \textbf{
r3
} & \textbf{
f3
} & \textbf{
onepps
} & \textbf{
gpstime
} & \textbf{
gpsdte
} & \textbf{
gps-valid
} & \textbf{
gps-satelites
} & \textbf{
xx
} & \textbf{
correction
}\\\hline

92328FE2
 & 
00
 & 
3D
 & 
00
 & 
3E
 & 
00
 & 
00
 & 
00
 & 
00
 & 
915E10CF
 & 
034016.021
 & 
060180
 & 
V
 & 
00
 & 
0
 & 
+0055
\\\hline
\end{tabulary}



\subsection{DAQ onboard documentation}
\label{hardware:daq-onboard-documentation}
Online help on the DAQ cards is available by sending the following commands to the DAQ
\begin{itemize}
\item {} 
V1, V2, V3

\item {} 
H1,H2

\end{itemize}


\subsubsection{V1}
\label{hardware:v1}
\begin{tabulary}{\linewidth}{|L|L|L|}
\hline
\textbf{
Setting
} & \textbf{
example value
} & \textbf{
description
}\\\hline

Run Mode
 & 
Off
 & 
CE (cnt enable), CD (cnt disable )
\\\hline

Ch(s) Enabled
 & 
3,2,1,0
 & 
Cmd DC  Reg C0 using (bits 3-0)
\\\hline

Veto Enable
 & 
Off
 & 
VE 0 (Off),  VE 1 (On)
\\\hline

Veto Select
 & 
Ch0
 & 
Cmd DC  Reg C0 using (bits 7,6)
\\\hline

Coincidence 1-4
 & 
1-Fold
 & 
Cmd DC  Reg C0 using (bits 5,4)
\\\hline

Pipe Line Delay
 & 
40 nS
 & 
Cmd DT  Reg T1=rDelay  Reg T2=wDelay  10nS/cnt
\\\hline

Gate Width
 & 
100 nS
 & 
Cmd DC  Reg C2=LowByte Reg C3=HighByte 10nS/cnt
\\\hline

Veto Width
 & 
0 nS
 & 
Cmd VG  (10nS/cnt)
\\\hline

Ch0 Threshold
Ch1 Threshold
Ch2 Threshold
Ch3 Threshold
 & 
0.200 vlts
0.200 vlts
0.200 vlts
0.200 vlts
 & \\\hline

Test Pulser Vlt
Test Pulse Ena
 & 
3.000 vlts
Off
 & \\\hline
\end{tabulary}


Example line for 1 of 4 channels. (Line Drawing, Not to Scale)
Input Pulse edges (begin/end) set rising/falling tags bits.
\_\_\_\_\textasciitilde{}\textasciitilde{}\textasciitilde{}\textasciitilde{}\textasciitilde{}\textasciitilde{}\_\_\_\_\_\_\_\_\_\_\_\_\_\_\_\_\_\_\_\_\_\_\_\_\_\_\_\_\_\_\_\_\_ Input Pulse, Gate cycle begins
\_\_\_\_\_\_\_\_\_\_\_\_\_\_\_\_\_\_\textasciitilde{}\_\_\_\_\_\_\_\_\_\_\_\_\_\_\_\_\_\_\_\_\_\_\_\_ Delayed Rise Edge `RE' Tag Bit
\_\_\_\_\_\_\_\_\_\_\_\_\_\_\_\_\_\_\_\_\_\_\_\_\textasciitilde{}\_\_\_\_\_\_\_\_\_\_\_\_\_\_\_\_\_\_ Delayed Fall Edge `FE' Tag Bit
\_\_\_\_\_\_\_\_\_\_\_\_\_                           Tag Bits delayed by PipeLnDly
\_\_\_\textbar{}        {\color{red}\bfseries{}\textbar{}\_\_\_\_\_\_\_\_\_\_\_\_\_\_\_\_\_\_\_\_\_\_\_\_\_ PipeLineDelay :   40nS
\_\_\_\_\_\_\_\_\_\_\_\_\_\_\_\_\_\_\_\_\_
\_\_\_\_\_\_\_\_\_\_\_\_\_\_\_\_\_\textbar{}}                     {\color{red}\bfseries{}\textbar{}\_\_\_ Capture Window:   60nS
\_\_\_\_\_\_\_\_\_\_\_\_\_\_\_\_\_\_\_\_\_\_\_\_\_\_\_\_\_\_\_\_\_\_\_
\_\_\_\textbar{}}                                   {\color{red}\bfseries{}\textbar{}}\_\_\_ Gate Width    :  100nS

If `RE','FE' are outside Capture Window, data tag bit(s) will be missing.
CaptureWindow = GateWidth - PipeLineDelay
The default Pipe Line Delay is 40nS, default Gate Width is 100nS.
Setup CMD sequence for Pipeline Delay.  CD,  WT 1 0, WT 2 nn (10nS/cnt)
Setup CMD sequence for Gate Width.  CD, WC 2 nn(10nS/cnt), WC 3 nn (2.56uS/cnt)

H2

Barometer      Qnet Help Page 2
BA      - Display Barometer trim setting in mVolts and pressure as mBar.
BA d    - Calibrate Barometer by adj. trim DAC ch in mVlts (0-4095mV).
Flash
FL p    - Load Flash with Altera binary file({\color{red}\bfseries{}*}.rbf), p=password.
FR      - Read FPGA setup flash, display sumcheck.
FMR p   - Read page 0-3FF(h), (264 bytes/page)
Page 100h= start fpga {\color{red}\bfseries{}*}.rbf file, page 0=saved setup.
GPS
NA 0    - Append NMEA GPS data Off,(include 1pps data).
NA 1    - Append NMEA GPS data On, (Adds GPS to output).
NA 2    - Append NMEA GPS data Off,(no 1pps data).
NM 0    - NMEA GPS display, Off, (default), GPS port speed 38400, locked.
NM 1    - NMEA GPS display (RMC + GGA + GSV) data.
NM 2    - NMEA GPS display (ALL) data, use with GPS display applications.
Test Pulser
TE m    - Enable run mode,  0=Off, 1=One cycle, 2=Continuous.
TD m    - Load sample trigger data list, 0=Reset, 1=Singles, 2=Majority.
TV m    - Voltage level at pulse DAC, 0-4095mV, TV=read.
Serial \#
SN p n  - Store serial \# to flash, p=password, n=(0-65535 BCD).
SN      - Display serial number (BCD).
Status
ST      - Send status line now.  This resets the minute timer.
ST 0    - Status line, disabled.
ST 1 m  - Send status line every (m) minutes.(m=1-30, def=5).
ST 2 m  - Include scalar data line, chs S0-S4 after each status line.
ST 3 m  - Include scalar data line, plus reset counters on each timeout.
TI n     - Timer (day hr:min:sec.msec), TI=display time, (TI n=0 clear).
U1 n     - Display Uart error counter, (U1 n=0 to zero counters).
VM 1     - View mode, 0x80=Event\_Demarcation\_Bit outputs a blank line.
- View mode returns to normal after `CD','CE','ST' or `RE'.

H1
Quarknet Scintillator Card,  Qnet2.5  Vers 1.11  Compiled Jul 15 2009  HE=Help
Serial\#=6531     uC\_Volts=3.33      GPS\_TempC=0.0     mBar=1023.8

CE     - TMC Counter Enable.
CD     - TMC Counter Disable.
DC     - Display Control Registers, (C0-C3).
WC a d - Write   Control Registers, addr(0-6) data byte(H).
DT     - Display TMC Reg, 0-3, (1=PipeLineDelayRd, 2=PipeLineDelayWr).
WT a d - Write   TMC Reg, addr(1,2) data byte(H), if a=4 write delay word.
DG     - Display GPS Info, Date, Time, Position and Status.
DS     - Display Scalar, channel(S0-S3), trigger(S4), time(S5).
RE     - Reset complete board to power up defaults.
RB     - Reset only the TMC and Counters.
SB p d - Set Baud,password, 1=19K, 2=38K, 3=57K ,4=115K, 5=230K, 6=460K, 7=920K
SA n   - Save setup, 0=(TMC disable), 1=(TMC enable), 2=(Restore Defaults).
TH     - Thermometer data display (@ GPS), -40 to 99 degrees C.
TL c d - Threshold Level, signal ch(0-3)(4=setAll), data(0-4095mV), TL=read.
Veto   - Veto select, Off='VE 0', On='VE 1', Gate='VG c', 0-255(D) 10ns/cnt.
View   - View setup registers. Setup=V1, Voltages(V2), GPS LOCK(V3).
HELP   - HE,H1=Page1, H2=Page2, HB=Barometer, HS=Status, HT=Trigger.

VE2
V2
Barometer Pressure Sensor
Calibration Voltage  = 1495 mVolts   Use Cmd `BA' to calibrate.
Sensor Output Voltage= 1655 mVolts   (2.93mV *  565 Cnts)
Pressure mBar        = 1023.6        (1655.5 - 1500)/15 + 1013.25
Pressure inch        = 30.63         (mBar / 33.42)

Timer Capture/Compare Channel
TempC  = 0.0     Error?  Check sensor cable connection at GPS unit.
TempF  = 32.0    (TempC * 1.8) + 32

Analog to Digital Converter Channels(ADC)
Vcc 1.80V = 1.82 vlts     (2.93mV *  621 Cnts)
Vcc 1.20V = 1.19 vlts     (2.93mV *  407 Cnts)
Pos 2.50V = 2.45 vlts     (2.93mV *  837 Cnts)
Neg 5.00V = 5.03 vlts     (7.38mV *  682 Cnts)
Vcc 3.30V = 3.33 vlts     (4.84mV *  689 Cnts)
Pos 5.00V = 4.84 vlts     (7.38mV *  656 Cnts)
5V Test    Max=4.86v    Min=4.84v    Noise=0.015v

V3
10 Second Accumulation of 1PPS Latched 25MHz Counter. (20 line buffer)
Buffer     Now (hex)     Prev-Now (dec) (25e6*10)
1              0               0
2              0               0
3              0               0
4              0               0
5              0               0
6              0               0
7              0               0
8              0               0
9              0               0
10              0               0
11              0               0
12              0               0
13              0               0
14              0               0
15              0               0
16              0               0
17              0               0
18              0               0
19              0               0
20              0               0


\section{muonic package software reference}
\label{muonic::doc}\label{muonic:muonic-package-software-reference}

\subsection{main package: muonic}
\label{muonic:module-muonic}\label{muonic:main-package-muonic}\index{muonic (module)}
{\hyperref[muonic:module-muonic.daq]{\code{muonic.daq}}}
{\hyperref[muonic:module-muonic.gui]{\code{muonic.gui}}}
{\hyperref[muonic:module-muonic.analysis]{\code{muonic.analysis}}}


\subsection{daq i/o with muonic.daq}
\label{muonic:module-muonic.daq}\label{muonic:daq-i-o-with-muonic-daq}\index{muonic.daq (module)}
Provide a connection to the QNet DAQ cards via python-serial. For software testing and development, (very) dumb DAQ card simulator is available


\subsubsection{\emph{muonic.daq.DAQProvider}}
\label{muonic:muonic-daq-daqprovider}
Control the two I/O threads which communicate with the DAQ. If the simulated DAQ is used, there is only one thread.
\phantomsection\label{muonic:module-muonic.daq.DAQProvider}\index{muonic.daq.DAQProvider (module)}\index{DAQProvider (class in muonic.daq.DAQProvider)}

\begin{fulllineitems}
\phantomsection\label{muonic:muonic.daq.DAQProvider.DAQProvider}\pysiglinewithargsret{\strong{class }\code{muonic.daq.DAQProvider.}\bfcode{DAQProvider}}{\emph{opts}, \emph{logger}, \emph{root}}{}
Launch the main part of the GUI and the worker threads. periodicCall and
endApplication could reside in the GUI part, but putting them here
means that you have all the thread controls in a single place.

\end{fulllineitems}



\subsubsection{\emph{muonic.daq.DAQConnection}}
\label{muonic:muonic-daq-daqconnection}
The module provides a class which uses python-serial to open a connection over the usb ports to the daq card. Since on LINUX systems the used usb device ( which is usually /dev/tty0 ) might change during runtime, this is catched automatically by DaqConnection. Therefore a shell script is invoked.
\phantomsection\label{muonic:module-muonic.daq.DaqConnection}\index{muonic.daq.DaqConnection (module)}\index{DaqConnection (class in muonic.daq.DaqConnection)}

\begin{fulllineitems}
\phantomsection\label{muonic:muonic.daq.DaqConnection.DaqConnection}\pysiglinewithargsret{\strong{class }\code{muonic.daq.DaqConnection.}\bfcode{DaqConnection}}{\emph{inqueue}, \emph{outqueue}, \emph{logger}}{}~\index{get\_port() (muonic.daq.DaqConnection.DaqConnection method)}

\begin{fulllineitems}
\phantomsection\label{muonic:muonic.daq.DaqConnection.DaqConnection.get_port}\pysiglinewithargsret{\bfcode{get\_port}}{}{}
check out which device (/dev/tty) is used for DAQ communication

\end{fulllineitems}

\index{read() (muonic.daq.DaqConnection.DaqConnection method)}

\begin{fulllineitems}
\phantomsection\label{muonic:muonic.daq.DaqConnection.DaqConnection.read}\pysiglinewithargsret{\bfcode{read}}{}{}
Get data from the DAQ. Read it from the provided Queue.

\end{fulllineitems}

\index{write() (muonic.daq.DaqConnection.DaqConnection method)}

\begin{fulllineitems}
\phantomsection\label{muonic:muonic.daq.DaqConnection.DaqConnection.write}\pysiglinewithargsret{\bfcode{write}}{}{}
Put messages from the inqueue which is filled by the DAQ

\end{fulllineitems}


\end{fulllineitems}



\subsubsection{\emph{muonic.daq.SimDaqConnection}}
\label{muonic:muonic-daq-simdaqconnection}
This module provides a dummy class which simulates DAQ I/O which is read from the file ``simdaq.txt''.
The simulation is only useful if the software-gui should be tested, but no DAQ card is available
\phantomsection\label{muonic:module-muonic.daq.SimDaqConnection}\index{muonic.daq.SimDaqConnection (module)}
Provides a simple DAQ card simulation, so that software can be tested
\index{SimDaq (class in muonic.daq.SimDaqConnection)}

\begin{fulllineitems}
\phantomsection\label{muonic:muonic.daq.SimDaqConnection.SimDaq}\pysiglinewithargsret{\strong{class }\code{muonic.daq.SimDaqConnection.}\bfcode{SimDaq}}{\emph{logger}}{}~\index{\_physics() (muonic.daq.SimDaqConnection.SimDaq method)}

\begin{fulllineitems}
\phantomsection\label{muonic:muonic.daq.SimDaqConnection.SimDaq._physics}\pysiglinewithargsret{\bfcode{\_physics}}{}{}
This routine will increase the scalars variables using predefined rates
Rates are drawn from Poisson distributions

\end{fulllineitems}

\index{inWaiting() (muonic.daq.SimDaqConnection.SimDaq method)}

\begin{fulllineitems}
\phantomsection\label{muonic:muonic.daq.SimDaqConnection.SimDaq.inWaiting}\pysiglinewithargsret{\bfcode{inWaiting}}{}{}
simulate a busy DAQ

\end{fulllineitems}

\index{readline() (muonic.daq.SimDaqConnection.SimDaq method)}

\begin{fulllineitems}
\phantomsection\label{muonic:muonic.daq.SimDaqConnection.SimDaq.readline}\pysiglinewithargsret{\bfcode{readline}}{}{}
read dummy pulses from the simdaq file till
the configured value is reached

\end{fulllineitems}

\index{write() (muonic.daq.SimDaqConnection.SimDaq method)}

\begin{fulllineitems}
\phantomsection\label{muonic:muonic.daq.SimDaqConnection.SimDaq.write}\pysiglinewithargsret{\bfcode{write}}{\emph{command}}{}
Trigger a simulated daq response with command

\end{fulllineitems}


\end{fulllineitems}

\index{SimDaqConnection (class in muonic.daq.SimDaqConnection)}

\begin{fulllineitems}
\phantomsection\label{muonic:muonic.daq.SimDaqConnection.SimDaqConnection}\pysiglinewithargsret{\strong{class }\code{muonic.daq.SimDaqConnection.}\bfcode{SimDaqConnection}}{\emph{inqueue}, \emph{outqueue}, \emph{logger}}{}~\index{read() (muonic.daq.SimDaqConnection.SimDaqConnection method)}

\begin{fulllineitems}
\phantomsection\label{muonic:muonic.daq.SimDaqConnection.SimDaqConnection.read}\pysiglinewithargsret{\bfcode{read}}{}{}
Simulate DAQ I/O

\end{fulllineitems}


\end{fulllineitems}



\subsection{pyqt4 gui with muonic.gui}
\label{muonic:pyqt4-gui-with-muonic-gui}
This package contains all gui relevant classes like dialogboxes and tabwidgets. Every item in the global menu is utilizes a ``Dialog'' class. The ``Canvas'' classes contain plot routines for displaying measurements in the TabWidget.
\phantomsection\label{muonic:module-muonic.gui}\index{muonic.gui (module)}
The gui of the programm, written with PyQt4


\subsubsection{\emph{muonic.gui.MainWindow}}
\label{muonic:muonic-gui-mainwindow}
Contains the  ``main'' gui application. It Provides the MainWindow, which initializes the different tabs and draws a menu.
\phantomsection\label{muonic:module-muonic.gui.MainWindow}\index{muonic.gui.MainWindow (module)}
Provides the main window for the gui part of muonic
\index{MainWindow (class in muonic.gui.MainWindow)}

\begin{fulllineitems}
\phantomsection\label{muonic:muonic.gui.MainWindow.MainWindow}\pysiglinewithargsret{\strong{class }\code{muonic.gui.MainWindow.}\bfcode{MainWindow}}{\emph{inqueue}, \emph{outqueue}, \emph{logger}, \emph{opts}, \emph{root}, \emph{win\_parent=None}}{}
The main application
\index{clear\_function() (muonic.gui.MainWindow.MainWindow method)}

\begin{fulllineitems}
\phantomsection\label{muonic:muonic.gui.MainWindow.MainWindow.clear_function}\pysiglinewithargsret{\bfcode{clear\_function}}{}{}
Reset the rate plot by clicking the restart button

\end{fulllineitems}

\index{closeEvent() (muonic.gui.MainWindow.MainWindow method)}

\begin{fulllineitems}
\phantomsection\label{muonic:muonic.gui.MainWindow.MainWindow.closeEvent}\pysiglinewithargsret{\bfcode{closeEvent}}{\emph{ev}}{}
Is triggered when the window is closed, we have to reimplement it
to provide our special needs for the case the program is ended.

\end{fulllineitems}

\index{config\_menu() (muonic.gui.MainWindow.MainWindow method)}

\begin{fulllineitems}
\phantomsection\label{muonic:muonic.gui.MainWindow.MainWindow.config_menu}\pysiglinewithargsret{\bfcode{config\_menu}}{}{}
Show the config dialog

\end{fulllineitems}

\index{create\_widgets() (muonic.gui.MainWindow.MainWindow method)}

\begin{fulllineitems}
\phantomsection\label{muonic:muonic.gui.MainWindow.MainWindow.create_widgets}\pysiglinewithargsret{\bfcode{create\_widgets}}{}{}
Initialize the tab widget

\end{fulllineitems}

\index{exit\_program() (muonic.gui.MainWindow.MainWindow method)}

\begin{fulllineitems}
\phantomsection\label{muonic:muonic.gui.MainWindow.MainWindow.exit_program}\pysiglinewithargsret{\bfcode{exit\_program}}{\emph{*args}}{}
This function is used either with the `x' button
(then an event has to be passed)
Or it is used with the File-\textgreater{}Exit button, than no event
will be passed.

\end{fulllineitems}

\index{help\_menu() (muonic.gui.MainWindow.MainWindow method)}

\begin{fulllineitems}
\phantomsection\label{muonic:muonic.gui.MainWindow.MainWindow.help_menu}\pysiglinewithargsret{\bfcode{help\_menu}}{}{}
Show a simple help menu

\end{fulllineitems}

\index{processIncoming() (muonic.gui.MainWindow.MainWindow method)}

\begin{fulllineitems}
\phantomsection\label{muonic:muonic.gui.MainWindow.MainWindow.processIncoming}\pysiglinewithargsret{\bfcode{processIncoming}}{}{}
Handle all the messages currently in the inqueue 
and parse the result to the corresponding widgets

\end{fulllineitems}

\index{threshold\_menu() (muonic.gui.MainWindow.MainWindow method)}

\begin{fulllineitems}
\phantomsection\label{muonic:muonic.gui.MainWindow.MainWindow.threshold_menu}\pysiglinewithargsret{\bfcode{threshold\_menu}}{}{}
Shows the threshold dialogue

\end{fulllineitems}


\end{fulllineitems}

\index{MuonicOptions (class in muonic.gui.MainWindow)}

\begin{fulllineitems}
\phantomsection\label{muonic:muonic.gui.MainWindow.MuonicOptions}\pysiglinewithargsret{\strong{class }\code{muonic.gui.MainWindow.}\bfcode{MuonicOptions}}{\emph{timewindow}, \emph{writepulses}, \emph{nostatus}, \emph{user}}{}
A simple struct which holds the different
options for the program

\end{fulllineitems}

\index{tr() (in module muonic.gui.MainWindow)}

\begin{fulllineitems}
\phantomsection\label{muonic:muonic.gui.MainWindow.tr}\pysiglinewithargsret{\code{muonic.gui.MainWindow.}\bfcode{tr}}{}{}
QCoreApplication.translate(str, str, str disambiguation=None, QCoreApplication.Encoding encoding=QCoreApplication.CodecForTr) -\textgreater{} QString
QCoreApplication.translate(str, str, str, QCoreApplication.Encoding, int) -\textgreater{} QString

\end{fulllineitems}



\subsubsection{\emph{muonic.gui.TabWidget}}
\label{muonic:muonic-gui-tabwidget}
This provides the interface to the different ``physics'' features of muonic, like a rate plot or a pulse display.
\phantomsection\label{muonic:module-muonic.gui.TabWidget}\index{muonic.gui.TabWidget (module)}
Manage the different (physics) widgets
\index{TabWidget (class in muonic.gui.TabWidget)}

\begin{fulllineitems}
\phantomsection\label{muonic:muonic.gui.TabWidget.TabWidget}\pysiglinewithargsret{\strong{class }\code{muonic.gui.TabWidget.}\bfcode{TabWidget}}{\emph{mainwindow}, \emph{timewindow}, \emph{logger}}{}
The TabWidget will provide a tabbed interface.
All functionality should be represented by tabs in the TabWidget
\index{activateMuondecayClicked() (muonic.gui.TabWidget.TabWidget method)}

\begin{fulllineitems}
\phantomsection\label{muonic:muonic.gui.TabWidget.TabWidget.activateMuondecayClicked}\pysiglinewithargsret{\bfcode{activateMuondecayClicked}}{}{}
What should be done if we are looking for mu-decays?

\end{fulllineitems}

\index{activatePulseanalyzerClicked() (muonic.gui.TabWidget.TabWidget method)}

\begin{fulllineitems}
\phantomsection\label{muonic:muonic.gui.TabWidget.TabWidget.activatePulseanalyzerClicked}\pysiglinewithargsret{\bfcode{activatePulseanalyzerClicked}}{}{}
set-up the pulseanalyzer widget

\end{fulllineitems}

\index{center() (muonic.gui.TabWidget.TabWidget method)}

\begin{fulllineitems}
\phantomsection\label{muonic:muonic.gui.TabWidget.TabWidget.center}\pysiglinewithargsret{\bfcode{center}}{}{}
\end{fulllineitems}

\index{mufitClicked() (muonic.gui.TabWidget.TabWidget method)}

\begin{fulllineitems}
\phantomsection\label{muonic:muonic.gui.TabWidget.TabWidget.mufitClicked}\pysiglinewithargsret{\bfcode{mufitClicked}}{}{}
fit the muon decay histogram

\end{fulllineitems}

\index{on\_file\_clicked() (muonic.gui.TabWidget.TabWidget method)}

\begin{fulllineitems}
\phantomsection\label{muonic:muonic.gui.TabWidget.TabWidget.on_file_clicked}\pysiglinewithargsret{\bfcode{on\_file\_clicked}}{}{}
save the raw daq data to a automatically named file

\end{fulllineitems}

\index{on\_hello\_clicked() (muonic.gui.TabWidget.TabWidget method)}

\begin{fulllineitems}
\phantomsection\label{muonic:muonic.gui.TabWidget.TabWidget.on_hello_clicked}\pysiglinewithargsret{\bfcode{on\_hello\_clicked}}{}{}
send a message to the daq

\end{fulllineitems}

\index{on\_periodic\_clicked() (muonic.gui.TabWidget.TabWidget method)}

\begin{fulllineitems}
\phantomsection\label{muonic:muonic.gui.TabWidget.TabWidget.on_periodic_clicked}\pysiglinewithargsret{\bfcode{on\_periodic\_clicked}}{}{}
issue a command periodically

\end{fulllineitems}

\index{startClicked() (muonic.gui.TabWidget.TabWidget method)}

\begin{fulllineitems}
\phantomsection\label{muonic:muonic.gui.TabWidget.TabWidget.startClicked}\pysiglinewithargsret{\bfcode{startClicked}}{}{}
restart the rate measurement

\end{fulllineitems}

\index{stopClicked() (muonic.gui.TabWidget.TabWidget method)}

\begin{fulllineitems}
\phantomsection\label{muonic:muonic.gui.TabWidget.TabWidget.stopClicked}\pysiglinewithargsret{\bfcode{stopClicked}}{}{}
hold the rate measurement plot till buttion is pushed again

\end{fulllineitems}

\index{timerEvent() (muonic.gui.TabWidget.TabWidget method)}

\begin{fulllineitems}
\phantomsection\label{muonic:muonic.gui.TabWidget.TabWidget.timerEvent}\pysiglinewithargsret{\bfcode{timerEvent}}{\emph{ev}}{}
Update the widgets

\end{fulllineitems}


\end{fulllineitems}

\index{tr() (in module muonic.gui.TabWidget)}

\begin{fulllineitems}
\phantomsection\label{muonic:muonic.gui.TabWidget.tr}\pysiglinewithargsret{\code{muonic.gui.TabWidget.}\bfcode{tr}}{}{}
QCoreApplication.translate(str, str, str disambiguation=None, QCoreApplication.Encoding encoding=QCoreApplication.CodecForTr) -\textgreater{} QString
QCoreApplication.translate(str, str, str, QCoreApplication.Encoding, int) -\textgreater{} QString

\end{fulllineitems}



\subsubsection{\emph{muonic.gui.ConfigDialog}}
\label{muonic:muonic-gui-configdialog}\label{muonic:module-muonic.gui.ConfigDialog}\index{muonic.gui.ConfigDialog (module)}\index{ConfigDialog (class in muonic.gui.ConfigDialog)}

\begin{fulllineitems}
\phantomsection\label{muonic:muonic.gui.ConfigDialog.ConfigDialog}\pysiglinewithargsret{\strong{class }\code{muonic.gui.ConfigDialog.}\bfcode{ConfigDialog}}{\emph{*args}}{}
\end{fulllineitems}



\subsubsection{\emph{muonic.gui.ThresholdDialog}}
\label{muonic:module-muonic.gui.ThresholdDialog}\label{muonic:muonic-gui-thresholddialog}\index{muonic.gui.ThresholdDialog (module)}\index{ThresholdDialog (class in muonic.gui.ThresholdDialog)}

\begin{fulllineitems}
\phantomsection\label{muonic:muonic.gui.ThresholdDialog.ThresholdDialog}\pysiglinewithargsret{\strong{class }\code{muonic.gui.ThresholdDialog.}\bfcode{ThresholdDialog}}{\emph{thr0}, \emph{thr1}, \emph{thr2}, \emph{thr3}, \emph{*args}}{}
\end{fulllineitems}



\subsubsection{\emph{muonic.gui.HelpDialog}}
\label{muonic:muonic-gui-helpdialog}\label{muonic:module-muonic.gui.HelpDialog}\index{muonic.gui.HelpDialog (module)}\index{HelpDialog (class in muonic.gui.HelpDialog)}

\begin{fulllineitems}
\phantomsection\label{muonic:muonic.gui.HelpDialog.HelpDialog}\pysiglinewithargsret{\strong{class }\code{muonic.gui.HelpDialog.}\bfcode{HelpDialog}}{\emph{*args}}{}~\index{helptext() (muonic.gui.HelpDialog.HelpDialog method)}

\begin{fulllineitems}
\phantomsection\label{muonic:muonic.gui.HelpDialog.HelpDialog.helptext}\pysiglinewithargsret{\bfcode{helptext}}{}{}
Show this text in the help window

\end{fulllineitems}


\end{fulllineitems}



\subsubsection{\emph{muonic.gui.PeriodicCallDialog}}
\label{muonic:module-muonic.gui.PeriodicCallDialog}\label{muonic:muonic-gui-periodiccalldialog}\index{muonic.gui.PeriodicCallDialog (module)}\index{PeriodicCallDialog (class in muonic.gui.PeriodicCallDialog)}

\begin{fulllineitems}
\phantomsection\label{muonic:muonic.gui.PeriodicCallDialog.PeriodicCallDialog}\pysiglinewithargsret{\strong{class }\code{muonic.gui.PeriodicCallDialog.}\bfcode{PeriodicCallDialog}}{\emph{*args}}{}
\end{fulllineitems}



\subsubsection{\emph{muonic.gui.PulseCanvas}}
\label{muonic:muonic-gui-pulsecanvas}\label{muonic:module-muonic.gui.PulseCanvas}\index{muonic.gui.PulseCanvas (module)}\index{PulseCanvas (class in muonic.gui.PulseCanvas)}

\begin{fulllineitems}
\phantomsection\label{muonic:muonic.gui.PulseCanvas.PulseCanvas}\pysiglinewithargsret{\strong{class }\code{muonic.gui.PulseCanvas.}\bfcode{PulseCanvas}}{\emph{parent}, \emph{logger}}{}
Matplotlib Figure widget to display Pulses
\index{color() (muonic.gui.PulseCanvas.PulseCanvas method)}

\begin{fulllineitems}
\phantomsection\label{muonic:muonic.gui.PulseCanvas.PulseCanvas.color}\pysiglinewithargsret{\bfcode{color}}{\emph{string}, \emph{color='none'}}{}
output colored strings on the terminal

\end{fulllineitems}

\index{update\_plot() (muonic.gui.PulseCanvas.PulseCanvas method)}

\begin{fulllineitems}
\phantomsection\label{muonic:muonic.gui.PulseCanvas.PulseCanvas.update_plot}\pysiglinewithargsret{\bfcode{update\_plot}}{\emph{pulses}}{}
\end{fulllineitems}


\end{fulllineitems}



\subsubsection{\emph{muonic.gui.ScalarsCanvas}}
\label{muonic:muonic-gui-scalarscanvas}\label{muonic:module-muonic.gui.ScalarsCanvas}\index{muonic.gui.ScalarsCanvas (module)}
Provide a canvas for a matplotlib rate plot
\index{ScalarsCanvas (class in muonic.gui.ScalarsCanvas)}

\begin{fulllineitems}
\phantomsection\label{muonic:muonic.gui.ScalarsCanvas.ScalarsCanvas}\pysiglinewithargsret{\strong{class }\code{muonic.gui.ScalarsCanvas.}\bfcode{ScalarsCanvas}}{\emph{parent}, \emph{logger}}{}
Matplotlib Figure widget to display Muon rates
\index{reset() (muonic.gui.ScalarsCanvas.ScalarsCanvas method)}

\begin{fulllineitems}
\phantomsection\label{muonic:muonic.gui.ScalarsCanvas.ScalarsCanvas.reset}\pysiglinewithargsret{\bfcode{reset}}{}{}
reseting all data

\end{fulllineitems}

\index{update\_plot() (muonic.gui.ScalarsCanvas.ScalarsCanvas method)}

\begin{fulllineitems}
\phantomsection\label{muonic:muonic.gui.ScalarsCanvas.ScalarsCanvas.update_plot}\pysiglinewithargsret{\bfcode{update\_plot}}{\emph{result}}{}
\end{fulllineitems}


\end{fulllineitems}



\subsubsection{\emph{muonic.gui.LifetimeCanvas}}
\label{muonic:muonic-gui-lifetimecanvas}\label{muonic:module-muonic.gui.LifetimeCanvas}\index{muonic.gui.LifetimeCanvas (module)}\index{LifetimeCanvas (class in muonic.gui.LifetimeCanvas)}

\begin{fulllineitems}
\phantomsection\label{muonic:muonic.gui.LifetimeCanvas.LifetimeCanvas}\pysiglinewithargsret{\strong{class }\code{muonic.gui.LifetimeCanvas.}\bfcode{LifetimeCanvas}}{\emph{parent}, \emph{logger}}{}
A simple histogram for the use with mu lifetime
measurement
\index{show\_fit() (muonic.gui.LifetimeCanvas.LifetimeCanvas method)}

\begin{fulllineitems}
\phantomsection\label{muonic:muonic.gui.LifetimeCanvas.LifetimeCanvas.show_fit}\pysiglinewithargsret{\bfcode{show\_fit}}{\emph{bin\_centers}, \emph{bincontent}, \emph{fitx}, \emph{decay}, \emph{p}, \emph{covar}, \emph{chisquare}, \emph{nbins}}{}
\end{fulllineitems}

\index{update\_plot() (muonic.gui.LifetimeCanvas.LifetimeCanvas method)}

\begin{fulllineitems}
\phantomsection\label{muonic:muonic.gui.LifetimeCanvas.LifetimeCanvas.update_plot}\pysiglinewithargsret{\bfcode{update\_plot}}{\emph{decaytimes}}{}
decaytimes must be a list of the last decays

\end{fulllineitems}


\end{fulllineitems}



\subsection{analyis package muonic.analysis}
\label{muonic:module-muonic.analysis}\label{muonic:analyis-package-muonic-analysis}\index{muonic.analysis (module)}

\subsubsection{\emph{muonic.analysis.PulseAnalyzer}}
\label{muonic:muonic-analysis-pulseanalyzer}
Transformation of ASCII DAQ data. Combination of Pulses to events, and looking for decaying muons with different trigger condi
\phantomsection\label{muonic:module-muonic.analysis.PulseAnalyzer}\index{muonic.analysis.PulseAnalyzer (module)}
Get the absolute timing of the pulses
by use of the gps time
Calculate also a non hex representation of
leading and falling edges of the pulses
\index{DecayTriggerThorough (class in muonic.analysis.PulseAnalyzer)}

\begin{fulllineitems}
\phantomsection\label{muonic:muonic.analysis.PulseAnalyzer.DecayTriggerThorough}\pysigline{\strong{class }\code{muonic.analysis.PulseAnalyzer.}\bfcode{DecayTriggerThorough}}
We demand a second pulse in the same channel where the muon got stuck
Should operate for a 10mu sec triggerwindow
\index{trigger() (muonic.analysis.PulseAnalyzer.DecayTriggerThorough method)}

\begin{fulllineitems}
\phantomsection\label{muonic:muonic.analysis.PulseAnalyzer.DecayTriggerThorough.trigger}\pysiglinewithargsret{\bfcode{trigger}}{\emph{triggerpulses}}{}
Hardcoded use of chan 1,2,3!!

\end{fulllineitems}


\end{fulllineitems}

\index{PulseExtractor (class in muonic.analysis.PulseAnalyzer)}

\begin{fulllineitems}
\phantomsection\label{muonic:muonic.analysis.PulseAnalyzer.PulseExtractor}\pysiglinewithargsret{\strong{class }\code{muonic.analysis.PulseAnalyzer.}\bfcode{PulseExtractor}}{\emph{pulsefile='`}}{}
get the pulses out of a daq line
speed is important here
\index{\_calculate\_edges() (muonic.analysis.PulseAnalyzer.PulseExtractor method)}

\begin{fulllineitems}
\phantomsection\label{muonic:muonic.analysis.PulseAnalyzer.PulseExtractor._calculate_edges}\pysiglinewithargsret{\bfcode{\_calculate\_edges}}{\emph{line}, \emph{counter\_diff=0}}{}
get the leading and falling edges of the pulses
Use counter diff for getting pulse times in subsequent 
lines of the triggerflag

\end{fulllineitems}

\index{\_get\_evt\_time() (muonic.analysis.PulseAnalyzer.PulseExtractor method)}

\begin{fulllineitems}
\phantomsection\label{muonic:muonic.analysis.PulseAnalyzer.PulseExtractor._get_evt_time}\pysiglinewithargsret{\bfcode{\_get\_evt\_time}}{\emph{time}, \emph{correction}, \emph{trigger\_count}, \emph{onepps}}{}
Get the absolute event time in seconds since day start
If gps is not available, only relative eventtime based on counts
is returned

\end{fulllineitems}

\index{\_order\_and\_cleanpulses() (muonic.analysis.PulseAnalyzer.PulseExtractor method)}

\begin{fulllineitems}
\phantomsection\label{muonic:muonic.analysis.PulseAnalyzer.PulseExtractor._order_and_cleanpulses}\pysiglinewithargsret{\bfcode{\_order\_and\_cleanpulses}}{}{}
Remove pulses which have a 
leading edge later in time than a 
falling edge and do a bit of sorting
Remove also single leading or falling edges
NEW: We add virtual falling edges!

\end{fulllineitems}

\index{close\_file() (muonic.analysis.PulseAnalyzer.PulseExtractor method)}

\begin{fulllineitems}
\phantomsection\label{muonic:muonic.analysis.PulseAnalyzer.PulseExtractor.close_file}\pysiglinewithargsret{\bfcode{close\_file}}{}{}
\end{fulllineitems}

\index{extract() (muonic.analysis.PulseAnalyzer.PulseExtractor method)}

\begin{fulllineitems}
\phantomsection\label{muonic:muonic.analysis.PulseAnalyzer.PulseExtractor.extract}\pysiglinewithargsret{\bfcode{extract}}{\emph{line}}{}
Analyze subsequent lines (one per call)
and check if pulses are related to triggers
For each new trigger,
return the set of pulses which belong to that trigger,
otherwise return None

\end{fulllineitems}


\end{fulllineitems}



\subsubsection{\emph{muonic.analysis.fit}}
\label{muonic:muonic-analysis-fit}
Provide a fitting routine
\phantomsection\label{muonic:module-muonic.analysis.fit}\index{muonic.analysis.fit (module)}
Script for performing a fit to a histogramm of recorded 
time differences for the use with QNet
\index{main() (in module muonic.analysis.fit)}

\begin{fulllineitems}
\phantomsection\label{muonic:muonic.analysis.fit.main}\pysiglinewithargsret{\code{muonic.analysis.fit.}\bfcode{main}}{\emph{bincontent=None}}{}
\end{fulllineitems}



\chapter{Indices and tables}
\label{index:indices-and-tables}\begin{itemize}
\item {} 
\emph{genindex}

\item {} 
\emph{modindex}

\item {} 
\emph{search}

\end{itemize}


\renewcommand{\indexname}{Python Module Index}
\begin{theindex}
\def\bigletter#1{{\Large\sffamily#1}\nopagebreak\vspace{1mm}}
\bigletter{m}
\item {\texttt{muonic}}, \pageref{muonic:module-muonic}
\item {\texttt{muonic.analysis}}, \pageref{muonic:module-muonic.analysis}
\item {\texttt{muonic.analysis.fit}}, \pageref{muonic:module-muonic.analysis.fit}
\item {\texttt{muonic.analysis.PulseAnalyzer}}, \pageref{muonic:module-muonic.analysis.PulseAnalyzer}
\item {\texttt{muonic.daq}}, \pageref{muonic:module-muonic.daq}
\item {\texttt{muonic.daq.DaqConnection}}, \pageref{muonic:module-muonic.daq.DaqConnection}
\item {\texttt{muonic.daq.DAQProvider}}, \pageref{muonic:module-muonic.daq.DAQProvider}
\item {\texttt{muonic.daq.SimDaqConnection}}, \pageref{muonic:module-muonic.daq.SimDaqConnection}
\item {\texttt{muonic.gui}}, \pageref{muonic:module-muonic.gui}
\item {\texttt{muonic.gui.ConfigDialog}}, \pageref{muonic:module-muonic.gui.ConfigDialog}
\item {\texttt{muonic.gui.HelpDialog}}, \pageref{muonic:module-muonic.gui.HelpDialog}
\item {\texttt{muonic.gui.LifetimeCanvas}}, \pageref{muonic:module-muonic.gui.LifetimeCanvas}
\item {\texttt{muonic.gui.MainWindow}}, \pageref{muonic:module-muonic.gui.MainWindow}
\item {\texttt{muonic.gui.PeriodicCallDialog}}, \pageref{muonic:module-muonic.gui.PeriodicCallDialog}
\item {\texttt{muonic.gui.PulseCanvas}}, \pageref{muonic:module-muonic.gui.PulseCanvas}
\item {\texttt{muonic.gui.ScalarsCanvas}}, \pageref{muonic:module-muonic.gui.ScalarsCanvas}
\item {\texttt{muonic.gui.TabWidget}}, \pageref{muonic:module-muonic.gui.TabWidget}
\item {\texttt{muonic.gui.ThresholdDialog}}, \pageref{muonic:module-muonic.gui.ThresholdDialog}
\end{theindex}

\renewcommand{\indexname}{Index}
\printindex
\end{document}
